\documentclass{article}

\usepackage[utf8]{inputenc}
\usepackage[english]{babel}
\usepackage{naproche}

\begin{document}
  \section{Russell's Paradox}

  Russell's paradox is the assertion that every set theory that contains an
  unrestricted comprehension principle leads to contradictions.
  In terms of \textit{sets} and \textit{classes} it can be phrased as follows.

  \begin{forthel}
    \begin{theorem}[Russell's Paradox]\label{russell}
      If every class is a set then we have a contradiction.
    \end{theorem}
    \begin{proof}
      Assume that every class is a set.
      Define \[ R = \class{x | \text{$x$ is a set such that $x \notin x$}}. \]
      Then $R$ is a set.
      Hence \[ R \in R \iff R \notin R. \]
      Contradiction.
    \end{proof}
  \end{forthel}
\end{document}
